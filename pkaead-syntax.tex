\section{Encryption Schemes}
\label{sec:prelims}
\label{sec:encryption}
We begin by ... [blah blah]

\paragraph{Notational conventions. }When $X,Y$ are bitstrings, we write $X \concat Y$ for their concatenation, $|X|$ for the bitlength, and $X\xor Y$ for their bitwise exclusive-or.  When $\calX$ is a set, we write $x \getsr \calX$ to mean that a value is sampled from $\calX$ and assigned to the variable~$x$.  Unless otherwise specified, sampling from sets is via the uniform distribution.

When~$F$ is an randomized algorithm, we write $x \getsr F(y_1,y_2,\cdots)$ to mean that~$F$ is run with the specified inputs and the result assigned to~$x$.  When~$F$ is determinsitic, we drop the $\$$-embellishment from the assignment arrow.  Algorithms may be provided access to one or more \emph{oracles}, and we write these as superscripts, e.g. $F^{\mathcal{O}}$; oracle access is black box and via an specified interface.  

An \emph{adversary} is a randomized algorithm.

\paragraph{Asymmetric (Public-key) AEAD Schemes. }
Fix sets $\pubkeys, \seckeys, \keydata, \adata, \metadata, \pubivs, \ptxts,
\ctxts$, the first two of which are nonempty.  An public-key AEAD
(PK-AEAD) scheme $\pkaead=(\Kgen,\Enc,\Dec)$ is a triple of algorithms.  The randomized \emph{key generation} algorithm~$\Kgen\colon\keydata\to\pubkeys\times\seckeys$ takes key-registration data $\kreg \in \keydata$ as input and returns a public-key, secret-key pair $(\pk,\sk)$.  We write $(\pk,\sk)\getsr\Kgen(\kreg)$ for the operation of key generation. 

The \emph{encryption} algorithm $\Enc \colon \pubkeys \times \adata \times \pubivs \times \metadata \times \ptxts \to \ctxts$ takes a public-key~$\pk\in\pubkeys$, public AD~$\header \in \adata$, initialization vector~$\pubiv \in \pubivs$, secret AD~$\seciv \in \metadata$ and a plaintext~$M \in \ptxts$, and returns a ciphertext~$C \in \ctxts$. 
%When $\metadata$ is empty, then encryption is randomized.  
%Otherwise, it is \emph{IV-based} and deterministic.
To stress the differing semantics of the inputs, key and non-private/private data, we write $\Encprim{\pk}{\header,\pubiv}{\seciv,M}$ instead of $\Enc(\pk,\header,\pubiv,\seciv,M)$ for the operation of encryption.  When encryption takes an oracle, we $\EncprimO{\pk}{\header,\pubiv}{\seciv,M}{\mathcal{O}}$ to clearly separate oracles from inputs.

We consider two kinds of deterministic decryption algorithms.  \emph{Decryption with secret-AD recovery} is 
a mapping $\Dec \colon \seckeys \times \adata \times \pubivs \times \ctxts \to \left(\metadata \times \ptxts\right) \cup \{\bot\}$.  It takes a secret-key~$\sk\in\seckeys$, associated data~$\header \in \adata$, public-IV~$\pubiv \in \pubivs$, and a ciphertext~$C \in \ctxts$, and returns a pair $(S,M) \in \metadata\times\ptxts$, or the distinguished symbol~$\bot$.  We write $(S,M) \gets \Decprim{\sk}{\header,\pubiv}{C}$ for the operation of decrpytion. 
For proper operation, we require that for all $\header\in\adata$, all $(\pk,\sk)\in\pubkeys\times\seckeys$ that may be output by $\Kgen(\header)$ with non-zero probability, and for all $\pubiv\in\pubivs$, $\seciv\in\secivs$, and $M\in\ptxts$, we have $\Decprim{\sk}{\header,\pubiv}{\Encprim{\pk}{\header,\pubiv}{S,M}}=(S,M)$.

\emph{Decryption without secret-AD recovery} is 
a mapping $\Dec \colon \seckeys \times \adata \times \pubivs \times \ctxts \to \ptxts \cup \{\bot\}$.  As before, we write $M \gets \Decprim{\sk}{\header,\pubiv}{C}$ for the operation of decrpytion.  For proper operation, we require that for all $\header\in\adata$, all $(\pk,\sk)\in\pubkeys\times\seckeys$ that may be output by $\Kgen(\kreg)$ with non-zero probability, and for all $\pubiv\in\pubivs$, $\seciv\in\secivs$, and $M\in\ptxts$, we have $\Decprim{\sk}{\header,\pubiv}{\Encprim{\pk}{\header,\pubiv}{S,M}}=M$.

\paragraph{Discussion. }  We have made various syntactic choices that both scope the applications we envision, and provide useful delinations for implementors.  Let us first explore them a bit.  

\begin{itemize}
\item\emph{Why two versions of decryption?}  We formalize both decryption that must recover the secret AD, and decryption that does not.  The former is interesting for applications in which the sender wants to bind non-message data to the plaintext that should be hidden from all but the receiver.  For example, the secret-AD~$\seciv$ might be the salted hash of a user password $H(\mathrm{salt} \concat \mathrm{pwd})$.  If this has been previously established, say by a user registering with a banking website, then allowing~$\seciv$ to be recovered admits authenticity checks of the form \emph{this plaintext was sent by this user}.  Secret-AD may also be sensitive plaintext \emph{metadata}, e.g.\ the ordinality of this particular plaintext within an application or protocol stream; the number of times this message has been sent; the identity of the sender on whose behalf this ciphertext was produced, and the identity of the intended recipient; the provenance of the data; etc. \tsnote{other things?}\tsnote{Keywrap -- see the Camenish, Chandran, Shoup paper.  They describe the role of labels in keywrap, but it isn't clear why all of that information should be \textit{public}.  Indeed, based on their discussion I think one can reasonably argue that some of it should be private.  They even say that one can implement labelled PKE by prepending the label to the plaintext (prior to encryption).  This wouldn't implement AD in the sense we typicallly have, anyway.  But it would implement our syntax with AD ``inside'' of encryption.}

When decryption is not permited by syntax to return the secret-AD, we envision that $\seciv$~may be something like an external (to encryption) source of variability.  For example, a counter or an entropic input provided by a layer ``above'' encryption in the network stack.  In combination with the public-IV~$\pubiv$ this allows a party at the application layer to protect itself against a badly implemented nonce generation wherever encryption is actually performed.  \tsnote{other things?}
This is the viewpoint taken by~\cite{mihir} which views nonce-based PKE as a method for hedging against bad randomness generation.

\item\emph{Why does decryption take $\pubiv,\header$ as input?} For applications like public-key-based keywrapping (and key transport?) the public-AD~$\header$ provides context to the receiver about the decrypted data. (See comments above)  Providing $\pubiv$ enables mechanisms for preventing replay.  \task{Examples?}\task{other things?}  The downside is that, in some settings, exisiting decryption APIs may need to be updated. \tsnote{This is why Mihir did not require decryption to take $\pubiv,\header$.}

\item\emph{In Mihir's paper the ``seed'' seems to play the same role as the secret AD...}  Mihir's paper encapulates the seed-production algorithm as part of the syntax of an encryption scheme.  We externalize this (see below), although it isn't clear there's a big difference here. Mihir's syntax for the encryption algorithm and ours are very simliar, essentially ours takes public-AD where his does not. \task{What application support does this provide that Mihir misses?  This is related to a task just above.}  

Looking ahead, our security notions are pretty different.  His definitions treat the seed (our~$\seciv$) as a generated-once (per lifetime of public-secret key pair) client-side secret.  This seed is used for every encryption, hence is independent of the message or any per-message context.  Thus our definitions treat the ``seed'' differently, and this motivates a different semantic (server-side secret vs.\ per-message metadata), even though the encryption algorithm syntax is essentially the same.  Note that our treatment of~$\seciv$ is a strict generalization of his, since (as we will see in a moment) the production of~$\seciv$ is stateful; this allows for a fixed value of~$\seciv$ after the initial state. \task{Mihir's nonce-generator and our secret-AD production algorithm have very similar APIs too.  Does he motivate his input~$X$, which he calls the ``nonce selector'' ($\nu$ in his notation).}

\item\emph{What's this key-registration data all about?} \tsnote{Letting $\Kgen$ take registration data as input allows for binding of AD to a $(\pk,\sk)$ pair.  This may support interesting things, like asserting that this ciphertext is produced using the same AD that was used to create a given public key.} \task{This needs thought.  I like the idea, but I don't know how ``real'' it is. Should this be part of this paper, or not? }
\end{itemize}
 

\paragraph{Secret-AD production. } To capture the notion that plaintexts (or the applications that produce them) may have secret-AD connected to them, we introduce the concept of a stateful \emph{secret-AD production} algorithm $\mdalg$.  On input state~$\st$ and plaintext~$M$, this algorithm returns secret-AD~$\seciv$ along with an updated state.  We write $(\st,S) \getsr \mdalg(\st,M)$.  For notational convenience, we assume that on input $\mdalg(\emptystring,\bot)$ the algorithm simply initializes its state.

We note that $\mdalg$ is not necessarily something that a pkAEAD-scheme designer would actually specify and implement.  We use $\mdalg$ to abstract away the process by which secret-AD~$S$ (e.g. plaintext metadata) is produced by the application that also produces the plaintext~$M$. \task{Give a couple of examples.}

Allowing $\mdalg$ to pass state to itself admits things like secret counters.  Additionally allowing $\mdalg$ to be randomized captures things such as entropic strings produced by applications or operating systems(?) and associated to the plaintext.\task{Real examples?}  It also allows for sampling of things like passwords or other user input, this sampling possibly controlled by the current state.  (For example, sampling a password only on the first encryption call.)


%\paragraph{Symmetric AEAD Schemes. } \tsnote{Fill in similarly, may ultimately consolidate.}


