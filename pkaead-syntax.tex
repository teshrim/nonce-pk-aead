\newcommand{\dimension}[1]{|| #1 ||}
\section{Encryption Schemes}
\label{sec:prelims}
\label{sec:encryption}
We begin by ... [blah blah]

\paragraph{Notational conventions. }When $X,Y$ are bitstrings, we write $X \concat Y$ for their concatenation, $|X|$ for the bitlength, and $X\xor Y$ for their bitwise exclusive-or.  When~$X,Y$ are arbitrary quantities, we write $\langle X,Y \rangle$ for some implict, invertible encoding of~$X,Y$ as a bitstring, and define $|X,Y|=|\langle X,Y\rangle|$.

When $\vec{X}=(X_1,X_2,\ldots,X_m)$ is a  vector, we write $\dimension{\vec{X}}$ for the number of components~$m$, and $|\vec{X}|=|X_1,X_2,\ldots,X_m|$.  We say that $\vec{X},\vec{Y}$ are \emph{length-equivalent} if $\dimension{\vec{X}}=\dimension{\vec{Y}}=m$ and, for all $i\in[m]$, $|X_i|=|Y_i|$.

When $\calX$ is a set, we write $x \getsr \calX$ to mean that a value is sampled from $\calX$ and assigned to the variable~$x$.  Unless otherwise specified, sampling from sets is via the uniform distribution.  
When~$F$ is an randomized algorithm, we write $x \getsr F(y_1,y_2,\cdots)$ to mean that~$F$ is run with the specified inputs and the result assigned to~$x$.  When~$F$ is determinsitic, we drop the $\$$-embellishment from the assignment arrow.  Algorithms may be provided access to one or more \emph{oracles}, and we write these as superscripts, e.g. $F^{\mathcal{O}}$; oracle access is black box and via an specified interface.  

An \emph{adversary} is a randomized algorithm.


\begin{table}[tp]
\centering
\begin{tabular}{c||c|c|c}
Input  & Privacy & Authenticity & Recoverability \\
\hline
&&&\\[-1.5ex]
%$\pubiv$   & No    & No         &  No \\
$\vad$      & never    & always     &  never \\
$\vmsg$   & always  & always    & always \\ 
$\umd$    & sometimes    & sometimes  &  never \\ [0.5ex]
\hline
\end{tabular}
\caption{Requirements of encryption inputs. ``Recoverability'' means that the indicated quantity is returned by decryption.  An entry of ``sometimes'' means that the indicated property may or may not be required by~$\umd$, depending on the context of its use.}
\end{table}

\paragraph{Deterministic Public-key Encryption with AD and auxiliary input. }
Fix sets $\pubkeys, \seckeys, \keydata, \adata, \ptxts,\ctxts$, the first two of which are nonempty.  An public-key AEAD
(PK-AEAD) scheme $\pkaead=(\Kgen,\Enc,\Dec)$ is a triple of algorithms.  The randomized \emph{key generation} algorithm~$\Kgen\colon\keydata\to\pubkeys\times\seckeys$ takes key-registration data $\kreg \in \keydata$ as input and returns a public-key, secret-key pair $(\pk,\sk)$.  We write $(\pk,\sk)\getsr\Kgen(\kreg)$ for the operation of key generation. 

The \emph{encryption} algorithm $\Enc \colon \pubkeys \times \adata^* \times \ptxts^* \times \umetadata \to \ctxts$ is a deterministic mapping.  It takes a public-key~$\pk\in\pubkeys$, vector-valued associated data~$\vad \in \adata^*$,  a vector-valued message~$\vmsg \in \ptxts^*$ and auxiliary data~$\umd \in \umetadata$, and returns a ciphertext~$C \in \ctxts$. 
To stress the differing semantics of the inputs, key and non-private/private data, we write $C\gets\Encprim{\pk}{\vad}{\vmsg,\umd} =\Enc(\pk,\vad, \vmsg,\umd)$ for the operation of encryption. 
\task{Assumptions on ciphertext lengths?} 


The \emph{decryption} algorithm $\Dec \colon \seckeys \times \adata^* \times \ctxts \times \bits^* \to \ptxts^* \cup \{\bot\}$ is a deterministic mapping.  It takes a secret-key~$\sk\in\seckeys$, vector-valued AD~$\vad \in \adata$, ciphertext~$C \in \ctxts$ and auxiliary input~$\auxinput\in\bits^*$, and returns a vector $\vmsg \in \ptxts^*$ or the distinguished symbol~$\bot$.  We write $\vmsg \gets \Decprim{\sk}{\vad}{C,\auxinput}$ for the operation of decryption.   

To define correctness, we demand that for each~$\umd\in\umetadata$ there exists a non-empty set of \emph{witnesses} $\mathcal{Z}_\umd \subseteq \bits^*$ such that:  for all $\vad\in\adata^*$, all $(\pk,\sk)\in\pubkeys\times\seckeys$ that may be output by $\Kgen(\kreg)$ with non-zero probability, and for all $\pubiv\in\pubivs$, $\auxinput\in\mathcal{Z}_\umd$,  and $M\in\ptxts$, we have $\Decprim{\sk}{\vad}{\Encprim{\pk}{\vad}{\vmsg,\umd},\auxinput}=\vmsg$. 

\paragraph{Discussion. } Perhaps the most significant syntactic addition that we make to standard deterministic PKE is the auxilary data, $\umd$ for encryption and $\auxinput$ for decryption.  We envision multiple uses for~$\umd$, and give three example scenarios here.

First, consider the case that the encrypting and decrpyting have a pre-shared secret~$\umd$.  Then $\mathcal{Z}_T=\{\umd\}$ suffices to support authenticity checks as part of decryption.  There are numerous other possibilities for witnesses, depending on how~$\umd$ is used during encryption.  For example, if~$F$ is a PRF and $F_\umd(V)$ is recoverable from the ciphertext (where~$V$ may be a constant, or a component of the vector-AD~$\vad$), then $\mathcal{Z}_T=\{F_\umd(V)\}$ suffices. \tsnote{Going to need to support this, to avoid the ``why use PKE at all if you have a shared secret?'' Here I'm thinking about PSK for session resumption, 0-RTT applications.} 

Second, consider the case that the encrypting party has a secret~$T$ that is not shared with the decrypting party, but the decrypting party \emph{does} have information sufficient to verify that the sender has~$T$.  For example when~$\umd=\mathrm{pwd}$ is a user password, and the decrypting party holds $\langle \mathrm{salt},H(\mathrm{salt} \concat \mathrm{pwd})\rangle$, the salted hash of the users password.  Here too authenticity will be achievable.

We have formalized encryption to be deterministic, and a line of results \cite{xxx,yyy,zzz} show that strong notions of privacy require a sufficient amount of min-entropy in $(\pk,\vad,\vmsg,\umd)$ in each call to encryption.  
So as our third example, $\umd$~can be used to inject entropy.  
Perhaps the application that produces~$\vmsg$ wants to protect itself against bad randomness at the point in the communication stack where encryption of~$\vmsg$ is actually performed.  In this case no authentication (based on~$\umd$) is demanded, and the set $\mathcal{Z}_T=\bits^*$ suffices.

We note two other departures from more standard syntax for deterministic PKE.  First, we provide encryption with a vector of associated data.  This allows for ``stealing'' a component to use as a nonce that, in conjunction with a secret~$\umd$, can be leveraged by constructions to provide per-encryption randomness.  We stress that the associated data~$\vad$ and the auxiliary input~$\umd$ do have different semantics.  The former must be provided by both the encrypting and decrypting parties, and our security notions will demand that if these parties provide different~$\vad$ then decryption will fail.  But considering the examples just given, it is not the case that~$\umd$ will be known to both parties; hence it cannot simply be considered as part of the associated data.

Finally, our syntax also allows for vector-valued messages.  Among other things, this supports the bundling together of data that is related but not necessarily processed in the same way.  For example, traditional private data (i.e., plaintext) and private metadata.  

 
\paragraph{Key-registration data. } \task{This needs thought.  I like the idea, but I don't know how ``real'' it is. Should this be part of this paper, or not? }Letting $\Kgen$ take registration data as input allows for binding of AD to a $(\pk,\sk)$ pair.  This may support interesting things, like asserting that this ciphertext is produced using the same AD that was used to create a given public key.  (Note that this seems related to so-called ``predicate encryption''.)


               %----------------------------------------- O L D  S Y N T A X  ---------------------------------------------%

%\paragraph{IV-based Public-key AD schemes. }
%Fix sets $\pubkeys, \seckeys, \keydata, \adata, \umetadata, \rmetadata, \pubivs, \ptxts,\ctxts$, the first two of which are nonempty.  An public-key AEAD(PK-AEAD) scheme $\pkaead=(\Kgen,\Enc,\Dec)$ is a triple of algorithms.  The randomized \emph{key generation} algorithm~$\Kgen\colon\keydata\to\pubkeys\times\seckeys$ takes key-registration data $\kreg \in \keydata$ as input and returns a public-key, secret-key pair $(\pk,\sk)$.  We write $(\pk,\sk)\getsr\Kgen(\kreg)$ for the operation of key generation. 

%The deterministic \emph{encryption} algorithm $\Enc \colon \pubkeys \times \adata \times \pubivs \times \left(\umetadata \times \rmetadata\right) \times \ptxts \to \ctxts$ takes a public-key~$\pk\in\pubkeys$, public AD~$\header \in \adata$, initialization vector~$\pubiv \in \pubivs$, secret AD~$(\umd,\rmd) \in \umetadata \times \rmetadata$ and a plaintext~$M \in \ptxts$, and returns a ciphertext~$C \in \ctxts$. 
%To stress the differing semantics of the inputs, key and non-private/private data, we write $C\gets\Encprim{\pk}{\header,\pubiv}{\umd,\rmd,M} =\Enc(\pk,\header,\pubiv,(\umd,\rmd),M)$ for the operation of encryption.  %What is shipped to the receiver is $\langle\header,\pubiv,C \rangle$. \task{Assumptions on ciphertext lengths?} 


%The deterministic \emph{decryption} algorithm $\Dec \colon \seckeys \times \adata \times \pubivs \times \ctxts \times \bits^* \to \left(\rmetadata \times \ptxts\right) \cup \{\bot\}$.  It takes a secret-key~$\sk\in\seckeys$, public AD~$\header \in \adata$, initial value~$\pubiv \in \pubivs$, ciphertext~$C \in \ctxts$ and auxiliary input~$\auxinput\in\bits^*$, and returns a pair $(\rmd,M) \in \rmetadata\times\ptxts$, or the distinguished symbol~$\bot$.  We write $(\rmd,M) \gets \Decprim{\sk}{\header,\pubiv}{C,\auxinput}$ for the operation of decryption.   For each~$\umd\in\umetadata$ we assume there exists a non-empty set $\mathcal{Z}_\umd \subseteq \bits^*$ such that:  for all $\header\in\adata$, all $(\pk,\sk)\in\pubkeys\times\seckeys$ that may be output by $\Kgen(\header)$ with non-zero probability, and for all $\pubiv\in\pubivs$, $\seciv\in\secivs$, $\auxinput\in\mathcal{Z}_\umd$,  and $M\in\ptxts$, we have $\Decprim{\sk}{\header,\pubiv}{\Encprim{\pk}{\header,\pubiv}{\umd,\rmd,M},\auxinput}=(\rmd,M)$. 
% \tsnote{The auxilary input~$\auxinput$ is meant to support authentication \textit{within} the decryption boundary.  For example, when~$\auxinput$ is something like the receiver-side stored version of~$S$. }

%We note that in a typically application, the encrypting party sends $\langle\header,\pubiv,C \rangle$ to the decrypting party. The latter provides the auxillary input~$\auxinput$ to the decryption call.

%We have made various syntactic choices that both scope the applications we envision, and provide useful delinations for implementors.  Let us first explore them a bit.  

