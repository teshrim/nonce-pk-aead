%%%%%%%%%%%%%%%%%%%%%%%%%%%%%%%%%%%%%%%%%

\section{Security notions for PK-AEAD} 
\subsection{Current notions}
Here we consider candidate security notions for an IV-based PK-AEAD scheme.  Fix a scheme $\pkaead=(\Kgen,\Enc,\Dec)$ with secret-AD space~$\metadata$, and a secret-AD generation algorithm $\mdalg$ outputting strings in~$\metadata$. \tsnote{Insert appropriate qualifications on $\mdalg$ once these are sorted out.} Let~$\advA$ be an adversary.  

In Figure~\ref{fig:pkaead-notions} we give a notion of privacy.  The adversary queries an oracle that, on input $(\header,\pubiv,M_0,M_1)$, returns the encrpytion of $(\seciv,M_b)$ where~$S$ is produced by executing $\mdalg(\st,M_b)$ with the its current state~$\st$, and~$b$ is the random challenge-bit.  In this notion, we assume that $|M_0|=|M_1|$ in every query.

As previously noted, if the adversary controls~$\header, \pubiv$ and the plaintext messages, and encryption is deterministic, then the notion in Figure~\ref{fig:pkaead-notions} is unachievable if the secret-AD~$\seciv$ is predictable. Thus we introduce an auxillary notion, that of \textit{predictability of the secret-AD}.  In this security experiment, the adversary sends plaintext strings~$X$ to an oracle~$\predOracle$ that executes $\mdalg(\st,X)$ on the current state~$\st$ and adds the resulting secret-AD~$\seciv$ to a set~$Q$.  After the adversary has finished loading~$Q$ via these queries, it attempts to guess an element of~$Q$. \tsnote{This formalism of unpredictable~$S$ may not be sufficient. Intuitively you want to say somthing like, if you can't predict~$S$ then ciphertexts should be secure when the adversary is nonce-respecting.  It isn't clear how you use this PRED-notion in a reduction, because the privacy game PRIV provides ciphertexts that depend on~$S$, whereas the PRED game provides no information about~$S$. (Maybe if you let the adversary specify the state in its queries, you could do the reduction... but that seems really strong, since $\mdalg$ is the abstract process by which metadata is created, which might include the operating system, other applications, hardware, etc.) It also isn't clear that PRED is strong enough.  You lose the PRED game if even one bit of every element of~$Q$ is unpredictable.  Is one bit of unpredictability going to suffice?  }




\begin{figure}
\begin{center}
\fpage{.5}{
 \hpagess{.425}{.55}{
 \underline{$\ExpINDCDA{\kreg,\mdalg,\pkaead}{\advA}$}:\\[2pt]
 $(\pk,\sk)\getsr\Kgen(\kreg)$\\
 $\st \getsr \mdalg(\emptystring,\bot)$\\
 $b\getsr\bits$\\
 $b'\getsr\advA^{\encOracle}(\pk)$\\
 Return $[b'=b]$\\

\medskip
 \underline{$\ExpPred{\mdalg}{\advP}$}:\\[2pt]
 $\st \getsr \mdalg(\emptystring,\bot)$\\
 $ Q \gets \emptyset$\\ 
 $ S'\getsr\advP^{\predOracle}$\\
 Return $[S' \in Q ]$

 }
 {
 \Oracle{$\encOracle(\header,\pubiv,M_0,M_1)$}:\\[2pt]
 $(\st,\seciv) \getsr \mdalg(\st,M_b)$\\
 Return $\Encprim{\pk}{\header,\pubiv}{S,M_b}$\\[40pt]

\medskip
 \Oracle{$\predOracle(X)$}:\\[2pt]
 $(\st,\seciv) \getsr \mdalg(\st,X)$\\
 $Q \gets Q \cup \{\seciv\}$\\

 }
}
\caption{{\bf Top:} Privacy notion for an IV-based PK-AEAD
  scheme~$\pkaead$.  Secret-AD is produced by~$\mdalg$.  The key-pair
  is produced using registration-data~$\kreg$. {\bf Bottom:}
  Unpredictability notion for secret-AD algorithm~$\mdalg$.} \task{Our
  PRED notion is essentially the same as Mihir's PRED notion.  How
  does he use it in his proofs?}
\label{fig:pkaead-notions}
\end{center}
\end{figure}

\if{0}
\begin{figure}
\begin{center}
\fpage{.5}{
 \hpagess{.425}{.55}{
 \underline{$\ExpINDCDA{\kreg,\mdalg,\pkaead}{\advA,\simulator}$}:\\[2pt]
 $(\pk,\sk)\getsr\Kgen(\kreg)$\\
 $\st \getsr \mdalg(\emptystring,\bot)$\\
 $\st' \gets \st$\\
 $b\getsr\bits$\\
 $b'\getsr\advA^{\encOracle}(\pk)$\\
 Return $[b'=b]$\\
 }
 {
 \Oracle{$\encOracle(\header,\pubiv,M_0,M_1)$}:\\[2pt]
 if $b=1$\\
 \nudge $(\st,\seciv) \getsr \mdalg(\st,M_b)$\\
 \nudge Return $\Encprim{\pk}{\header,\pubiv}{S,M_b}$\\
 else \\
 \nudge $(\st', S) \getsr\simulator(\st', |M_0|)$\\
 \nudge Return $\Encprim{\pk}{\header,\pubiv}{S,M_b}$\\
 }
}
\caption{Another potential notion, this one simulation-based.}
\label{fig:pkaead-notions-2}
\end{center}
\end{figure}
\fi

\subsection{Older notions}
Here we consider candidate security notions for an IV-based PK-AEAD scheme.  Fix a scheme $\pkaead=(\Kgen,\Enc,\Dec)$ and a randomized algorithm~$\advD$ that samples from the secret-IV set~$\secivs$ (of $\pkaead$) according to some distibution.  Let~$\mu_\advD$ be the min-entropy of this distribution.  Let~$\advA$ be an adversary.  

In Figure~\ref{fig:ind-cda} we give two notions of plaintext privacy.  In the first, the adversary queries an oracle that, on input $(\header,\pubiv,M_0,M_1)$, returns the encrpytion of $(\seciv,M_b)$ where~$S$ was previously sampled by~$\advD$, and~$b$ is the random challenge-bit.  In this notion, we assume that $|M_0|=|M_1|$ in every query and that~$\advA$ is \emph{nonce-respecting}, meaning that it never repeats a value of~$\pubiv$ across its queries.  We refer to this as the Priv security notion.
In the second notion, the adversary does not control the value of~$\pubiv$.  Instead, when~$\advA$ queries $(\header,M_0,M_1)$, a fresh public-IV~$\pubiv$ is sampled from the public-IV set~$\pubivs$.  The oracle returns~$\pubiv$ along with the encryption of $(S,M_b)$ using that particular public-IV.  We refer to this as Priv-IV security notion.

In practice, security with respect to the first notion means that the PK-AEAD scheme will protect plaintexts so long as the calling environment provides nonces for the IVs, e.g. a reliable counter value.  Security with respect to the second notion means that the PK-AEAD scheme guarantees privacy only when the calling environment provides IVs that are random and independent across calls.

\tsnote{There are other verions, too, that might make sense.  In one, $S\getsr\advD$ not once, but each time the oracle is queried.  There's no need for nonce respecting behavior then.  In another, the adversary can decide whether or not a new~$\seciv$ is sampled on a query, and it must be nonce-respecting for a given~$\seciv$.  An ``$\seciv$-adaptive version of this would allow~$\advA$ to refer to sampled values of~$\seciv$ by handles and arbitrarly interleave queries to handles.}

For these two notions, we define corresponding advantage measures for a given adversary~$\advA$, sampler~$\advD$, and scheme $\pkaead$ as
\begin{align*}
\AdvINDCDA{\pkaead}{\advD,\advA}&=2\Prob{\ExpINDCDA{\pkaead}{\advD,\advA}=1}-1\\ \AdvINDCDAR{\pkaead}{\advD,\advA}&=2\Prob{\ExpINDCDAR{\pkaead}{\advD,\advA}=1}-1
\end{align*}
respectively.  In both experiments, we track the following adversarial resources: (1) the time-complexity~$t$ of~$\advA$, relative to some understood model of computation; (2) the query-complexity~$q$, measured as the number of queries made by~$\advA$ to its oracle; (3) the total query-length~$\sigma$, defined the be the sum over all query lengths, where $|(\header,\pubiv,M_0,M_1)|=|\header|+|\pubiv|+|M_0|$.
\begin{figure}
\begin{center}
\fpage{.5}{
 \hpagess{.425}{.55}{
 \underline{$\ExpINDCDA{\pkaead}{\advD,\advA}$}:\\[2pt]
 $(\pk,\sk)\getsr\Kgen$\\
 $b\getsr\bits$\\
 $S\getsr\advD$\\
 $b'\getsr\advA^{\encOracle(\cdot,\cdot,\cdot,\cdot)}(\pk)$\\
 Return $[b'=b]$\\

\medskip
 \underline{$\ExpINDCDAR{\pkaead}{\advD,\advA}$}:\\[2pt]
 $(\pk,\sk)\getsr\Kgen$\\
 $b\getsr\bits$\\
 $S\getsr\advD$\\
 $b'\getsr\advA^{\encOracle(\cdot,\cdot,\cdot)}(\pk)$\\
 Return $[b'=b]$

 }
 {
 \Oracle{$\encOracle(\header,\pubiv,M_0,M_1)$}:\\[2pt]
 Return $\Encprim{\pk}{\header,\pubiv}{S,M_b}$\\[51pt]

\medskip
 \Oracle{$\encOracle(\header,M_0,M_1)$}:\\[2pt]
 $N \getsr \pubivs$\\
 $C \gets  \Encprim{\pk}{\header,\pubiv}{S,M_b}$\\
 Return $(N,C)$

 }
}
\caption{Privacy notions for an IV-based PK-AEAD scheme~$\pkaead$ with public-IV and secret-IV sets $\pubivs,\secivs$ (resp.)}
\label{fig:ind-cda}
\tsnote{Doesn't match current syntax, where $\Kgen$ takes AD as input.  Need to think about how to handle this in the security notion(s).}
\end{center}
\end{figure}


\tsnote{Need authenticity notions.}