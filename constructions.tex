\section{Constructions}
\label{sec:constructions}
In Figure~\ref{fig:ro-kem-dem-construction} we give a simple construction of an IV-based PK-AEAD scheme that is secure  in the random-oracle model. \task{Prove, and nail down assumptions.}  It works for any randomized encapsulation algorithm $\OLencapprim{\pk}{}$, replacing its internally generated randomness with the output of the random oracle.  For example, the standard El Gamal KEM where $K=g^{\overline{S}x}$ and $X=g^{\overline{S}}$, for a generator~$g$ and $(\pk,\sk)=(g^x,x)$.


\begin{figure}[tbhp]
\begin{center}
\fpage{.5}{
 \hpagess{.45}{.55}{
 \underline{$\Encprim{\pk}{\header,\pubiv}{\umd,\rmd,M}$}:\\[2pt]
 $R \gets H(\pk \concat \header \concat \pubiv \concat \umd \concat \rmd )$\\ 
 $(K,\wrap) \gets \encapprim{\pk}{\header,\pubiv}{R}$\\
 $C \gets \encprim{K}{\header,\pubiv}{\langle \umd,\rmd,M\rangle}$\\
 Return $(\wrap,C)$
 }
 {
 \underline{$\Decprim{\sk}{\header,\pubiv}{\wrap,C,\auxinput}$}:\\[2pt]
 $K \gets \decapprim{\sk}{\header,\pubiv}{\wrap}$\\[3pt]
 if $\decprim{K}{\header,\pubiv}{C} = \bot$ then\\
 \hspace*{3ex} Return $\bot$\\
 $\langle \umd,\rmd,M \rangle \gets \decprim{K}{\header,\pubiv}{C}$\\
 if $\auxinput \in \mathcal{Z}_\umd$ then \\
 \nudge Return $(\rmd, M)$\\
 Return $\bot$
 }
}

\medskip
\hspace*{.5ex}\fpage{.5}{
 \hpagess{.45}{.55}{
 \underline{$\encapprim{\pk}{\header,\pubiv}{R}$}:\\[2pt]
 $(K,X) \gets \OLencapprim{\pk}{}{R}$\\
 Return $(K,X)$
 }
 {
 \underline{$\decapprim{\sk}{\header,\pubiv}{X}$}:\\[2pt]
 $K \gets \OLdecapprim{\sk}{}{X}$\\
 Return $K$
 } 
} 
\caption{An IV-based PK-AEAD scheme, built in the KEM-DEM style,  that is secure in the random-oracle model for hash function~$H$. (To be proved.)  The unspecified encapsulation algorithm $\overline{\encap}$ can be any randomized algorithm with its internal randomness replaced by~$\overline{\seciv}$.  We assume there is an efficient test for membership in the set $\mathcal{Z}_\umd$.
}
\label{fig:ro-kem-dem-construction}
\end{center}
\end{figure}



%This particular construction allows for recovery of~$\overline{S}$ upon decryption.  In turn, this allows a re-encryption authenticty check on the decapsulated key~$K$, in addition to any authenticity checks on the ciphertext~$C$ that the AEAD scheme imposes.  \tsnote{The auth check on~$K$ is indirect, and we'll need to nail down what properties of the AEAD scheme are needed.}  
%By having decryption return $M \concat \overline{S}$, we might enable higher-level checks.  For example, if the senders secret IV~$S$ has been previously stored receiver-side, or transmitted out-of-band.  
%Note that the secret IV is not required to change across encryption calls, so $S=\mbox{password}$ and $M=\mbox{ptxt}\concat \mbox{hash}(\mbox{password}\concat\mbox{salt})$ may provide a way for higher-level applications to check authenticity of the user, assuming $\mbox{hash}(\mbox{password}\concat\mbox{salt})$ is already known receiver-side.
%\tsnote{Need to patch up PKE syntax and proper operation requirement, if decryption returns something other than the plaintext that was encrypted (or $\bot$)} 

