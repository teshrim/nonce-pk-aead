%%%%%%%%%%%%%%%%%%%%%%%%%%%%%%%%%%%%%%%%%%%%%%%%%%%%%%%%%%%%%%%%%%%%%%%%%%%%%%%%
\section{Message Indistinguishability}

\tnote{The below is basically a scratchpad, you should probably ignore it for now...}

Returning to the intuition given above, our goal 
for HE security is to ensure that trial decryptions result in plausible 
honeymessages. This should hold \emph{even} for low-entropy keys and 
messages, setting HE security apart from more traditional goals.
We start with the following security game, which we call 
message indistinguishability (MI). It asks that an HE scheme ensure that,
relative to some message distribution $\mdist$  and key distribution $\kdist$,
an attacker cannot distinguish between a challenge message $\msg^*$ drawn from $\mdist$
and a message produced by decrypting an honest encryption of $\msg^*$ using an
adversarially-specified key. See \figref{fig:mi-sec}. The advantage of an adversary
$\advA$ in winning the MI game is defined as 
\bnm
  \AdvMI{\HEscheme,\mdist,\kdist}(\advA) = 2\cdotsm\Prob{\MI_{\HEscheme,\mdist,\kdist}^\advA \Rightarrow \true} - 1
\enm
where the probability is over the coins used by the security game and the event corresponds
to the game returning $\true$. 

\begin{figure}[t]
\center
\hpages{.22}{
\fpage{.97}{
\underline{$\MR_{\HEscheme,\mdist,\kdist}^\advA$}\\[2pt]
$\key^* \getk \kspace \next \msg^* \getm \mspace$\\
$\ctxt^* \getsr \encHE(\key^*,\msg^*)$\\
$\msg \getsr \advA(\ctxt^*)$\\
ret $\msg = \msg^*$
}
\fpage{.97}{
\underline{$\KR_{\HEscheme,\mdist,\kdist}^\advA$}\\[2pt]
$\key^* \getk \kspace \next \msg^* \getm \mspace$\\
$\ctxt^* \getsr \encHE(\key^*,\msg^*)$\\
$\key \getsr \advA(\ctxt^*)$\\
ret $\key = \key^*$
}
}{
\fpage{.97}{
\underline{$\MI_{\HEscheme,\mdist,\kdist}^\advA$}\\[2pt]
$b \getsr \zo$\\
$\key^* \getk \kspace \next \msg_1 \getm \mspace$\\
$\ctxt^* \getsr \encHE(\key^*,\msg_1)$\\
$b' \getsr \advA^\Chall(\ctxt^*)$\\
ret $b' = b$\medskip

\underline{proc.~$\Chall(\key)$}\\[2pt]
$\msg_0 \getsr \decHE(\key,\ctxt^*)$\\
ret $\msg_b$
}
}
\caption{Security game for message recovery (top left), key recovery (bottom left),
and message indistinguishability (right).}
\label{fig:mi-sec}
\end{figure}


We also define in \figref{fig:mi-sec} games for message recovery and 
key recovery, and define advantage in the normal way for each:
$\AdvMR{\HEscheme,\mdist,\kdist}(\advA) = \Pr[\MR_{\HEscheme,\mdist,\kdist}^\advA\Rightarrow \true]$
and 
$\AdvKR{\HEscheme,\mdist,\kdist}(\advA) = \Pr[\KR_{\HEscheme,\mdist,\kdist}^\advA\Rightarrow \true]$.
Note that unless $\mdist$ specifies a single message, these are unknown message 
attacks, which is weaker than normal mechanisms which allow known or chosen message attacks.


We start with the following proposition, which establishes that MI security
implies MR security. 

\begin{theorem}
Fix distributions $\mdist$, $\kdist$, and HE scheme $\HEscheme$. Let $\advA$ be an MR adversary
running in time $t$. Then for the MI adversary $\advB$ given in the proof below it holds that
\bnm
  \AdvMR{\HEscheme,\mdist,\kdist}(\advA) \le 2\cdotsm\AdvMI{\HEscheme,\mdist,\kdist}(\advB)
\enm
$\advB$ runs in time that of $\advA$ and makes a single query. 
\end{theorem}

\begin{proof}
Adversary $\advB$ operates as follows. On input $\ctxt^*$, it runs $\advA(\ctxt^*)$, which
in turn outputs a message guess $M$.  Then $\advB$ queries $\Chall(K)$ for an 
arbitrary $K$ and is given back a message $M'$. If $M = M'$ then $\advB$ outputs 1 and 
otherwise outputs a random bit. In the case that the MI challenge bit $b$ is 1, we
have that $\advB$ correctly outputs 1 at least as often as when $\advA$ succeeds, 
meaning $\Pr[\MI1_{\HEscheme,\mdist,\kdist}^\advB\Rightarrow \true] 
  \ge \Pr[\MR_{\HEscheme,\mdist,\kdist}^\advA\Rightarrow\true]$ 
where $\MI1$ is the  $\MI$ game except with bit set to 1. 

\end{proof}


