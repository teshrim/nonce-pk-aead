\section{Encryption Schemes}
\label{sec:prelims}
\label{sec:encryption}
We begin by ... [blah blah]

\paragraph{Notational conventions. }When $X,Y$ are bitstrings, we write $X \concat Y$ for their concatenation, $|X|$ for the bitlength, and $X\xor Y$ for their bitwise exclusive-or.  When $\calX$ is a set, we write $x \getsr \calX$ to mean that a value is sampled from $\calX$ and assigned to the variable~$x$.  Unless otherwise specified, sampling from sets is via the uniform distribution.

When~$F$ is an randomized algorithm, we write $x \getsr F(y_1,y_2,\cdots)$ to mean that~$F$ is run with the specified inputs and the result assigned to~$x$.  When~$F$ is determinsitic, we drop the $\$$-embellishment from the assignment arrow.  Algorithms may be provided access to one or more \emph{oracles}, and we write these as superscripts, e.g. $F^{\mathcal{O}}$; oracle access is black box and via an specified interface.  

An \emph{adversary} is a randomized algorithm.

\paragraph{Asymmetric (Public-key) Encryption Schemes. }
Fix sets $\pubkeys, \seckeys, \adata, \pubivs, \secivs, \ptxts, \ctxts$, the first two of which are nonempty.  An encryption scheme $\pkscheme=(\Kgen,\Enc,\Dec)$ is a triple of algorithms.  The randomized \emph{key generation} algorithm~$\Kgen$ takes no input and returns a public-key, secret-key pair $(\pk,\sk)$.  We write $(\pk,\sk)\getsr\Kgen$ for the operation of key generation.

The \emph{encryption} algorithm $\Enc \colon \pubkeys \times \adata \times \pubivs \times \secivs \times \ptxts \to \ctxts$ takes a public-key~$\pk\in\pubkeys$, associated data~$\header \in \adata$, public-IV~$\pubiv \in \pubivs$, secret-IV~$\seciv \in \secivs$ and a plaintext~$M \in \ptxts$, and returns a ciphertext~$C \in \ctxts$. 
When $\secivs$ is empty, then encryption is randomized.  Otherwise, it is \emph{IV-based} and deteministic.
To stress the differing semantics of the inputs, key and non-private/private data, we write $\Encprim{\pk}{\header,\pubiv}{S,M}$ instead of $\Enc(\pk,\header,\pubiv,\seciv,M)$ for the operation of encryption.  When encryption takes an oracle, we $\EncprimO{\pk}{\header,\pubiv}{\seciv,M}{\mathcal{O}}$ to clearly separate oracles from inputs.

The deterministic \emph{decryption} algorithm $\Dec \colon \seckeys \times \adata \times \pubivs \times \ctxts \to \left(\secivs \times \ptxts\right) \cup \{\bot\}$ takes a secret-key~$\sk\in\seckeys$, associated data~$\header \in \adata$, public-IV~$\pubiv \in \pubivs$, and a ciphertext~$C \in \ctxts$, and returns a pair $(S,M) \in \secivs\times\ptxts$, or the distinguished symbol~$\bot \not\in \ptxts$.  We write $(S,M) \gets \Decprim{\sk}{\header,\pubiv}{C}$ for the operation of decrpytion. 

For proper operation, we require that for all $(\pk,\sk)\in\pubkeys\times\seckeys$, $\header\in\adata$, $\pubiv\in\pubivs$, $\seciv\in\secivs$, and $M\in\ptxts$, we have $\Decprim{\sk}{\header,\pubiv}{\Encprim{\pk}{\header,\pubiv}{S,M}}=M$.


\paragraph{Symmetric Encryption Schemes. } \tsnote{Fill in similarly, may ultimately consolidate.}


\paragraph{Security notions. } 
\begin{figure}
\begin{center}
\fpage{.5}{
 \hpagess{.425}{.55}{
 \underline{$\ExpINDCDA{\pkscheme,\advD}{\advA}$}:\\[2pt]
 $(\pk,\sk)\getsr\Kgen$\\
 $b\getsr\bits$\\
 $S\getsr\advD$\\
 $b'\getsr\advA^{\OEnc(\cdot,\cdot,\cdot,\cdot)}(\pk)$\\
 Return $[b'=b]$\\

\medskip
 \underline{$\ExpINDCDAR{\pkscheme,\advD}{\advA}$}:\\[2pt]
 $(\pk,\sk)\getsr\Kgen$\\
 $b\getsr\bits$\\
 $S\getsr\advD$\\
 $b'\getsr\advA^{\OEnc(\cdot,\cdot,\cdot)}(\pk)$\\
 Return $[b'=b]$

 }
 {
 \Oracle{$\OEnc(\header,\pubiv,M_0,M_1)$}:\\[2pt]
 Return $\encprim{\pk}{\header,\pubiv}{S,M_b}$\\[54pt]

\medskip
 \Oracle{$\OEnc(\header,M_0,M_1)$}:\\[2pt]
 $N \getsr \pubivs$\\
 $C \gets  \encprim{\pk}{\header,\pubiv}{S,M_b})$\\
 Return $(N,C)$

 }
}
\caption{\textbf{Above:} The IND-CDA notion for IV-based PK-AEAD scheme~$\pkscheme$
  when the secret-IV sampler is~$\advD$.  We assume~$\advA$ is
  nonce-respecting. \textbf{Below:} The IND-CDA-IV notion, in which
  the public-IV~$\pubiv$ is a per-message random value.}
\label{fig:ind-cda}
\end{center}
\end{figure}
