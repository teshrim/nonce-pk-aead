\section{Proofs}
\label{sec:proofs}

\begin{theorem}
Let $pke$ be a (possibly) randomized public-key encryption scheme, $pkAEAD$ be the hash-then-encrypt 
$pkAEAD$ scheme, and $XD$ be a secret nonce generator.  Then, for any adversary $\advA$, the advantage of $\advA$ is bounded
by an adversary $\advB$ breaking the semantic security of $pke$ and a predictor $\advP$ breaking the predictability of $XD$ as:
\bne
\Adv^{Priv}_{U,XD,pkAEAD}(\advA) \leq \Adv^{IND-CPA}_{pke}(\advB) + 2q_h\Adv^{Pred}_{XD}(\advP)
\ene
When $\advA$ makes $q_e$ queries to its encryption oracle and $q_h$ queries to its random oracle, $\advB$ makes $q_e$ queries to its
encryption oracle and $q_e + q_h$ queries to its random oracle, and $\advP$ makes $q_e$ queries to the prediction oracle and $q_e + q_h$
 queries to the random oracle. 
\end{theorem}

\begin{figure}[tbhp]
\begin{center}
\fpage{.5}{
\hpagess{.45}{.55}{
 \experimentv{$\Exp^{\mathrm{IND-CPA}}_{\Pi,m}{(\advB)}$:}\\
 $\pk = \pk_{chal}$\\
 $st_0 \getsr XD(\epsilon,\bot)$\\
 $st_1 = st_0$\\
 $T = \emptyset$\\
 $b' \getsr \advA^{Enc,H}(\pk)$\\
Return $b'$\\ 
 }
 {
  \queryl{$H(x)$}\\
  $Z \getsr Coins$\\
  if $T[x] \neq \bot:$\\
  \nudge BAD $= \true$\\
  \nudge $Z = T[x]$\\
  $T[x] = Z$\\
  Return $Z$\\
 }
 \hpagess{.55}{.45}{
\queryl{$Enc(A,m_0,m_1)$}\\
$(st_0, T_0) \getsr XD(st_0,m_0)$\\
$(st_1, T_1) \getsr XD(st_1,m_1)$\\
$R_0 \getsr H(\langle pk,A,M_0,T_0\rangle)$\\
$R_1 \getsr H(\langle pk,A,M_1,T_1\rangle)$\\
Return $\enc_{pk}(\langle M_0,T_0\rangle, \langle M_1,T_1\rangle)$\\
 }
 {

 }
}
\caption{Adversary running against the IND-CPA experiment }
\label{fig:CPA}
\end{center}
\end{figure}

\begin{figure}[tbhp]
\begin{center}
\fpage{.5}{
\hpagess{.45}{.55}{
 \experimentv{$\Exp^{\mathrm{Pred}}_{XD}{(\advP)}$:}\\
 $(\pk,\sk) \getsr \kspace$\\
 $b \getsr \bits$\\
 $\cal X = \emptyset$\\
  $\cal Q = \emptyset$\\
 $b' \getsr \advA^{Enc,H}(\pk)$\\
 $t^* \getsr \cal X \cap \cal Q$\\
Return $t^*$\\ 
 }
 {
  \queryl{$H(x)$}\\
  $Z \getsr Coins$\\
  $\cal X = \cal X \cup x$ \\
  Return $Z$\\
 }
 \hpagess{.55}{.45}{
\queryl{$Enc(A,m_0,m_1)$}\\
$XD(m_b)$\\
$T \gets ???$\\
$q = \langle pk,A,m_b,T\rangle$\\
$\cal Q = \cal Q \cup q$\\
$R \getsr H(q)$\\
Return $\enc_{pk}(\langle m_b,T\rangle;R)$\\
 }
 {

 }
}
\caption{Adversary running against the Pred experiment}
\label{fig:PRED}
\end{center}
\end{figure}

\begin{proof}
Given adversary $\advA$ playing $\Exp^{Priv}_{U,XD,pkAEAD}$, $\advB$ runs as in Figure~\ref{fig:CPA}.  Here, the set $Coins$ is defined
by the set of all possible random coins used in the $pke$ encryption scheme.  Note also that we prevent $\advA$ from sending the same
query to $Enc(\cdot,\cdot,\cdot)$ twice by assuming that one component of $A$ is used as a nonce that does not repeat.  Given this
restriction, the $BAD$ flag will only be set if $\advA$ queries the random oracle on a value that was previously submitted to the encryption
oracle, which requires $\advA$ to guess the $T$ value associated with one of the messages it queried.  As long as
$BAD$ remains false, each query to $Enc$ is correctly simulated with fresh randomness drawn from the $IND-CPA$ encryption oracle,
and each query to $H$ is correctly simulated with fresh randomness drawn from the random oracle.  Thus, if $BAD = \false$, then
$\advB$ has the same advantage in the $IND-CPA$ experiment as $\advA$ has in the $Priv$ experiment.

If $BAD = \true$, then $\advA$ may distinguish between the fresh randomness used by the $IND-CPA$ oracle and the simulated
random oracle maintained by $\advB$.  So, to bound the probability of $\advA$ guessing any $T$ value, we design a prediction adversary $\advP$ that runs as shown in Figure~\ref{fig:PRED} in the prediction game $\Exp^{pred}_XD$ as defined in~\cite{BellareTackmann}.  If the intersection between queried messages and hash queries from $\advA$ is non-empty, it
implies that $\advA$ has guessed one of the $T$ values stored in the $pred$ oracle.  Since $\advP$ possesses the set of queried $T$ values 
in $\cal X$ but cannot determine which to return, it has a $\frac{1}{q_h}$ probability of guessing and returning the predicted $T$ value and winning
the prediction game.  Thus, we arrive at the following bound:

\bea
Pr[\Exp^{Priv}_{U,XD,pkAEAD}(\advA) = 1]  = &Pr[\Exp^{IND-CPA}_{pke}(\advB) = 1 | !BAD]Pr[!BAD]\\
 + &Pr[\Exp^{IND-CPA}_{pke}(\advB) = 1 | BAD]Pr[BAD]\\
\leq &Pr[\Exp^{IND-CPA}_{pke}(\advB) = 1|!BAD] + Pr[BAD]\\
\leq  &Pr[\Exp^{IND-CPA}_{pke}(\advB) = 1] + q_h\Adv^{Pred}_{XD}(\advP)\\
\Adv^{Priv}_{U,XD,pkAEAD}(\advA) \leq &\Adv^{IND-CPA}_{pke}(\advB) + 2q_h\Adv^{Pred}_{XD}(\advP)
\eea

Where the first inequality is a result of the universal theorem of game playing, the second is the worst-case bound on the ability of $\advA$ to
guess a secret metadata value $T$, and the final is from multiplying both sides by $2$ and subtracting $1$.  
Note that this proof assumes the worst-case that an adversary controls messages and associated data, so
the only unknown input to the random oracle is the value $T$.  Any added entropy in the joint distribution of these values tightens this bound.
\end{proof}


%Sample code---------------
%\begin{figure}[t]
%\noindent{$\mathcal{B}(PK)$}
%\begin{algorithmic}[1]
%\STATE $St = \emptyset$ 
%\STATE $PK = c \leftarrow e(X,Y)$
%\STATE $C = \emptyset$
%\STATE $V = \emptyset$
%\STATE $\text{run }\mathcal{A}(PK)$
%\end{algorithmic}
%
%\noindent{\bf On query} $Enc(A,M_0,M_1)$
%\begin{algorithmic}[1]
%\STATE $St', T_0 \leftarrow XD(St, M_0)$
%\STATE $St, T_1 \leftarrow XD(St, M_1)$
%\STATE $M'_0 = M_0||T_0$
%\STATE $M'_1 = M_1||T_1$
%%\IF {$M'_0||M'_1 \notin C$}
%\STATE{$C_b \leftarrow LR(M_0,T_0;M_1,T_1)$}
%\STATE $C = C \cup (M'_0,M'_1,C_b)$
%%\ELSE
%\STATE $C(M'_0||M'_1)$
%\RETURN $C_b$
%\end{algorithmic}
%
%\noindent{\bf On query} $H(x)$
%\begin{algorithmic}[1]
%\RETURN $H(x)$
%\end{algorithmic}
%\caption{Adversary}
%\label{fig:}
%\end{figure}