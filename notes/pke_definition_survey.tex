\documentclass[11pt, pdftex]{article}
\usepackage{epsf}
\usepackage{epsfig}
\usepackage{times}
\usepackage{ifthen}
\usepackage{comment}
\usepackage{amsmath, amsthm, amssymb}
\usepackage[margin=1in]{geometry}


\title{Deterministic Public Key Encryption}
\author{}
\date{}


\begin{document}
\maketitle

\section{Definitions}

\begin{enumerate}
\item Deterministic and Efficiently Searchable Encryption (BBO, Crypto 2007)
\begin{itemize}
\item Priv security ensures that a static adversary cannot generate side information about an underlying plaintext.  RO model construction.
\item requires high min-entropy in the plaintext space
\item ESE application and construction
\item CCA extension
\end{itemize}

\item Deterministic Encryption : Definitional Equivalences and Constructions without Random Oracles Relating the Security Notions (BeFiO, Crypto 2008)
\begin{itemize}
\item 7 security defs: (IND, 3 computational, 3 simulation based) against static adversaries.  All equivalent.  Standard model construction from TDP when plaintexts are UNIFORM/INDEPENDENT.  Extending this scheme requires stronger assumptions about the underlying TDP.
\item requires high min-entropy in the plaintext space
\item Build a randomized KEM from deterministic PK scheme
\end{itemize}


\item On notions of security for deterministic encryption, and efficient constructions without random oracles (BoFeO, Crypto 2008)
\begin{itemize}
\item PRIV-security for block sources, which assumes that for each message, the worst-case conditional min-entropy is at least some minimum value $t$
\item Equivalence to PRIV1 (and a single message IND notion).
\item Standard model constructions using ``lossy'' TDFs in both CPA and CCA settings
\item DDH-based and Pallier-based instantiations of the general construction, secure in the standard model.
\item KEM construction also provided
\end{itemize}

\item Hedged public-key encryption: How to protect against bad randomness (BBNRSSY, Asiacrypt 2009)
\begin{itemize}
\item IND-CDA security, where the joint distrib between the randomness and the message space has sufficiently high min-entropy
\item Assumes conditional min-entropy (like block sources) to achieve security
\item Adaptive source queries are allowed, but NOT after the PK has been revealed
\item Two RO constructions from IND-CPA schemes, two standard schemes from IND-CPA and PRIV schemes for block sources (using LTDFs)
\end{itemize}

\item Deterministic public-key encryption for adaptively chosen plaintext distributions (RSV, Eurocrypt 13)
\begin{itemize}
\item Adaptive security definition that generalizes across both IND-security and Hedged definitions
\item Adversary is GIVEN ACCESS to the public key and makes adaptive queries giving plaintext distributions rather than samples
\item Real-or-random game selects messages according to either the queried distribution (real) or uniform (random)
\item Two conditions on queries: each distribution has a min-entropy of $k$ (lower bounded) and the number of queried distributions are bounded by some polynomial w.r.t. the security parameter.
\item Build on the technique of BoFeO (using LTDFs) to yield a new standard model scheme that meets this adaptive definition.
\item CCA extension
\end{itemize}
\end{enumerate}

\section{Primitives}

\begin{enumerate}
\item Lossy Trapdoor Functions and Their Applications (PW, STOC 2008)
\begin{itemize}
\item New LTDF primitive requires that a TDF family be able to operate in two modes: lossy and injective
\item Security proven when both modes are indistinguishable
\item Instantiated under many assumptions (DDH applied later to D-PKE)
\end{itemize}

\item Instantiating Random Oracles via UCEs (BHK, CRYPTO 2013)
\begin{itemize}
\item Develops a notion of second-degree assumptions, which prove security against multi-stage adversaries
\item Capable of moving some random oracles into the standard model
\end{itemize}
\end{enumerate}

\section{Constructions}

\begin{enumerate}
\item Resisting Randomness Subversion : Fast Deterministic and Hedged Public-key Encryption in the Standard Model (BH, Eurocrypt 15)
\begin{itemize}
\item No new definitions.
\item Establishes the first standard model IND-secure D-PKE and H-PKE based on a combination of UCE for statistically uniform sources and LTDFs
\item Develop fast alternatives that yield KEM-DEM like efficiency
\item Generic U-PKE from D-PKE
\end{itemize}
\end{enumerate}

\section{Nonce-based PKE Summary}

This work introduces the concept of nonce-based public-key encryption as a means to defend against random number generator bugs or
subversion.  A nonce-based PKE scheme is made up of four algorithms:
\begin{align*}
NPE.Kg &\rightarrow (ek,dk)\\
NPE.sKg &\rightarrow (ak)\\
NPE.Enc(ek, ak, m, n) &\rightarrow c\\
NPE.Dec(dk, c) &\rightarrow m
\end{align*}
It is important to note that the seed $ak$ is not recoverable in decryption, and that the nonce is only used to help randomize the output
of $NPE.Enc$.  It is NOT sent with the message and is NOT needed for decryption.  This allows for an easier requirement on the random
number generator, that it simply return nonces instead of cryptographically random seeds.  However, the security maintained is equivalent to
IND-CCA security.

The main technique used to achieve this construction is a hedged extractor, an algorithm defined as $HE(ak, (m,n)) = r$.  An HE must specifically
meet two requirements: it is a PRF if $ak$ is secret, and is a randomness extractor when $ak$ is revealed but $(m,n)$ is unpredictable.  They formalize
the idea of unpredictability in nonces using a nonce generator, which can be manipulated indirectly by an adversary input, but is not directly controlled
by the adversary.  They then define security for the HE using a ``real or random'' game where an adversary may choose input to the HE,
but should not be able to distinguish HE output from a returned random string.  They also provide two unpredictability definitions for the underlying
nonce generator, which has implications on the security of the PKE scheme defined later.  It is important to note that the adversary in the
real or random game when $ak$ is revealed must be separated into two stages: one who accesses an RoR oracle and one who 
is given the seed $ak$ and state from the oracle phase and must guess the challenge bit.  They conclude their discussion of HE with a random oracle
construction and a standard model construction using a PRF and an ``almost-XOR-universal'' hash function.

The authors define nonce-based PKE security in a similar way to CCA security.  In one game, the secret $ak$ is chosen and hidden, and the
adversary is given $Enc$ and $Dec$ oracles and must distinguish real message ciphertexts from all-zero message ciphertexts when the nonce generator
does not repeat values.  The second game
is the same except that the secret $ak$ is revealed and the nonce generator is unpredictable.  
They provide a construction for achieving both definitions simultaneously by substituting the
random coins for a standard IND-CCA PKE scheme with the output of a hedged extractor.

The work concludes with unforgeability definitions for nonce-based signatures.  These follow the patterns established by the encryption
definitions, and the constructions function in the same way, requiring the underlying PKE signature scheme to be randomized so
that the coins can be produced by a HE.\\

\noindent PROS:
\begin{enumerate}
\item Provides a nice solution to randomness subversion
\item Does not require $Dec()$ to be modified
\item Demonstrates loosely how nonces can be applied more generally than just encryption
\end{enumerate}

\noindent CONS:
\begin{enumerate}
\item Does not allow nonce to be used for anything besides randomization since the receiver does not see it.
\item Trivially meets CPA and CCA security based on the underlying scheme, but does not consider true authenticated encryption.
This could seemingly be achieved by combining the signatures also developed in the work.
\item Does not really address why deterministic signature schemes are not ``good enough''.  Why do we need nonce-based randomized schemes
that require state to be maintained?
\item The secret $ak$ is only useful (like the nonce) for generating randomness.  Some applications could use it as a password or
authenticating value.
\item Specifically motivated to solve randomness concerns.  Not designed to augment any specific applications.
\end{enumerate}

\section{Public Key Authenticated Encryption (PKAE)}

\subsection{Syntax}
PKAE, also called signcryption in related work, was first formalized by An in 2001.  She defines a PKAE scheme as five algorithms:
\begin{itemize}
\item $CommonKeyGen(\lambda) \$\rightarrow I$
\item $SenderKeyGen(I) \$\rightarrow (PK_s, SK_s)$
\item $ReceiverKeyGen(I) \$\rightarrow (PK_r, SK_r)$
\item $Enc(SK_s, PK_s, PK_r, m) \$\rightarrow C$
\item $Dec(SK_r, C) \rightarrow (pk_s, M) \text{ or } \perp$
\end{itemize}

\subsection{Security Definitions}
An defines six notions of security for PKAE, two for confidentiality, four for unforgeability.
\begin{itemize}
\item{\bf Privacy:} The author uses the standard notions for confidentiality in the public key setting, IND-CPA and IND-CCA security.
\item{\bf Third-person (outsider) unforgeability:} The authors uses definitions that parallel the symmetric setting for outsider unforgeability.
Essentially, TUF-CTXT allows an adversary access to an encryption oracle, and the adversary must forge a valid ciphertext that has never
been returned by the oracle.  TUF-PTXT, by contrast, allows an adversary access to an encryption oracle, but must return a ciphertext which
decrypts to a message that was never sent to the oracle.
\item{\bf Receiver (insider) unforgeability:} In this definition, the receiver is the adversary.  It consists of two stages: the first is given the global
information $I$ and may choose the receiver keys.  The second phase is given access to an encryption oracle that encrypts messages to the
public key produced in phase one.  It has access to all public information, and must create a forged, never-before-seen ciphertext (CTXT) or
message (PTXT) as with the previous definition.  The second stage adversary also has the ability to \emph{change} the adversary key pair mid
experiment, simulating a receiver who revokes and changes their keys and is trying to forge a message based on previously received ciphertexts.
The author points out that this attack could be thwarted with timestamps, but still allows an adversary the ability to modify its keys.
\end{itemize}

\subsection{Example Construction}
The author examines three generic combinations of an IND-CPA encryption scheme and an unforgeable (both strongly and weakly) signature scheme
 (encrypt-and-sign, sign-then-encrypt, encrypt-then-sign) and shows that none of them
meet all the proposed definitions of security.  She then presents the following generic construction that meets all definitions of security.

$Enc'(PK_r, SK_s, PK_s, M):$
\begin{itemize}
\item $C' \leftarrow Enc_{PK_r}(M||PK_s)$
\item $\sigma \leftarrow Sign_{SK_s}(C'||PK_r)$
\item $C \leftarrow PK_s||C'||\sigma$\\
\end{itemize}

$Dec'(SK_r, C):$
\begin{itemize}
\item parse $C \leftarrow PK_s||C'||\sigma$
\item produce $PK_r$ based on $SK_r$
\item If $Verify_{PK_s}(C'||PK_r) = 0$ return $\perp$
\item $M||PK' \leftarrow Dec_{SK_r}(C')$
\item If $PK' = PK_s$ return $M$ else return $\perp$
\end{itemize}


\section{Questions}
\begin{enumerate}
\item What is the hierarchy of syntax?  Can we express H-PKE as IV-PKE? 
\item Are there IV-PKE schemes that fall outside of H-PKE schemes?
\item Hedged definition: what's the use of picking two messages?
\item ACD security: how does limiting the number of distributions prohibit dependence?
\end{enumerate}

\end{document}