\section{Related Work}
\label{rel_wk}
\paragraph{Summary of BT~\cite{BellareTackmann}. }

Bellare and Tackmann developed the first definition for nonce-based public key encryption~\cite{BellareTackmann}.
Their syntax and security definitions are designed to provide stronger security in the face of compromised random number generators
which would cause standard PKE and hedged PKE to fail.  They define nonce-based public key encryption (NPE) as a tuple
of four algorithms: key generation $NPE.Kg \rightarrow (ek,dk)$ produces a public and private key pair; seed generation
$NPE.sKg \rightarrow (xk)$ produces a uniformly random seed that is used across many encryptions and is known only to
the sender; encryption $NPE.Enc(ek, xk, m, n) \rightarrow c$ takes a public key, seed, message, and nonce, and outputs
a ciphertext; and decryption $NPE.Dec(dk, c) \rightarrow m$ takes a private key and ciphertext and returns the underlying
encrypted message.  Within this definition, two key points must be observed.  First, the seed $xk$ used in the encryption
process is not recoverable during decryption, and must remain secret to the sender alone.  Second, unlike symmetric key 
nonces, the nonce is not transmitted with the message and is not needed to decrypt the ciphertext.  This stems from a distinction
identified by the authors between nonces in the public and symmetric key settings, specifically, that some amount of hidden
randomness is required in either the seed or the nonce, or an adversary may trivially break the security of the scheme under 
any reasonable definition.

The authors propose two definitions for security, both of which must be met to be considered a secure NPE scheme.  The first
(NBP1) requires that IND-CCA security be achieved when the seed $xk$ is secret and the nonce $n$ is non-repeating across
all queries.  The second definition (NBP2) requires that IND-CCA security be achieved when the seed $xk$ is known by the
adversary and each nonce $n$ is unguessable.  This unguessable nonce requirement is formalized using a nonce generator
$NG$, which can be influenced by the adversary and must produce nonces that cannot be predicted based on knowledge of
initial state or prior output from the $NG$.  The authors then instantiate an NPE meeting these security definitions using a novel 
hedged extractor construction, which essentially functions as both a PRF (when the seed is hidden) and a randomness
extractor (when the seed is known).  Combining the random coins produced by this primitive with a randomized IND-CCA
secure PKE scheme produces an NPE scheme that meets both NBP1 and NBP2 security.  The authors conclude the work by
developing similar definitions and constructions for nonce-based signatures.  While not explicitly considered in the work,
these signatures could presumably be combined with NPE schemes to achieve authenticated encryption using nonces.

While this construction provides resilience against subverted randomness and a foundational definition for nonce-based PKE,
it leaves several nuanced applications to be considered in future work, and does not provide a truly general definition for nonce-based PKE.
The authors analyze security for IND-CCA secure encryption schemes, but do not consider authenticated encryption, even
with the development of nonce-based signatures.  Their NPE syntax requires that the nonce remain hidden for all messages,
and incorporates nonce generation security assumptions within the definition of the NPE.  In network applications such as 
the I2P anonymity network~\cite{zantout2011}, it may be possible to re-use packet counters or sequence numbers as nonces, which would
require a definition that includes public nonce values.  Furthermore, keeping the seed $xk$ secret to the sender alone and
making it non-recoverable precludes a variety of applications that could use a recoverable secret
seed as hidden application metadata or as an authenticating token such as a hashed password.  Our work seeks to develop a more general
definition for nonce-based PKE which augments the security of real networked applications.

\paragraph{Deterministic PKE. }
The first attempt to define public key security for fully deterministic schemes was by Bellare, Boldyreva, and O'Neill~\cite{BBO07}.
This work formalized a new definition of security that was achievable in the random oracle model as long as the min-entropy
of the plaintext space remained above a certain threshold.  Several other related definitions were proposed shortly after~\cite{BFOR08,BSO08},
producing instantiations in both the random oracle and the standard model, as well as constructions for semantically secure randomized schemes
and randomized key encapsulation mechanisms based on deterministic public key encryption.
Among all of these definitions, two major requirements persist: some restriction on the min-entropy of the plaintexts is always necessary, and
adversary access to the public key must be restricted through multi-phase adversaries. Raghunathan et al.~\cite{RSV13} proposed
a definition that relaxes this second restriction, but changes the adversary's goal to distinguishing between plaintext distributions
rather than individual plaintext queries.  While these deterministic definitions of security provide some insight into what is
required to maintain security in the presence of no randomness, they fail to demonstrate the potential for maintaining some
security when the randomness provided to the scheme is somewhere in between uniform and deterministic.  To answer
this question, hedged encryption schemes present definitions of security and encryption schemes that maintain some
amount of security even in the face of degrading quality randomness.  This is commonly achieved through harvesting randomness
from both the random coins provided to the encryption scheme and the plaintext distribution itself~\cite{BBNRTSSHY2009}.  
Recent work by Bellare and Hoang~\cite{BH15}
has produced standard model constructions for both deterministic and hedged encryption schemes by applying Universal Computational Extractors,
or UCEs~\cite{BHK13}.  This work also demonstrates the continuing utility of lossy trapdoor functions~\cite{PW11}, first applied to deterministic encryption
schemes by Boldyreva et al.~\cite{BSO08}.  Unfortunately, none of these definitions consider the potential for achieving equivalent security to 
semantic security given limited access to randomness, and can be viewed as special cases of nonce-based encryption by varying the
properties of the nonce.

\paragraph{Password based encryption. }
Password-based encryption has been in use for decades, but unfortunately the theoretical underpinnings that demonstrate the
security of common practices have lagged behind.  Several studies of specific password hashing protocols have been produced~\cite{LR88,WG00,Boyen07},
but unfortunately the low entropy of most passwords chosen in practice does not match the assumptions made by these
theoretical analyses in demonstrating security.  More recently, significant effort has been put into developing more general definitions
of security by concretely defining security for key derivation functions.  While the security of general KDFs using common hashing techniques have been
analyzed~\cite{DGHKR04}, the password-based KDF presents particular challenges due to the significantly lower entropy available from
passwords that can be memorized~\cite{Kraw10}.  Password-specific KDF security definitions have focused on defining indistinguishability from
a randomly chosen key~\cite{YY05}, while others have defined security against recovery of the password~\cite{KSHW98,AW05}.  Bellare et al.~\cite{BRT12}
expand on these definitions by developing a multi-instance definition of security for a KDF, which ensures that breaking the
security of a single password does not significantly improve an attacker's ability to learn additional passwords.
Beyond password-based KDFs, password authenticated key exchange (PAKE) protocols allow for two parties in possession of a password
to negotiate a cryptographic session key~\cite{BPR00,GL03,CHKLM05}.  However, the combination of added randomness being locally generated and the interactive
nature of these protocols eliminates many of the security issues associated with the offline dictionary attacks that are the primary concern for
password-based encryption.

\paragraph{Labeled PKE. }
The labeled public key encryption scheme~\cite{Shoup2001} was developed to allow for formal analysis of the authenticated metadata that often accompanies ciphertexts in practice.
Syntactically similar to the symmetric key extension called associated data~\cite{Rog02}, a label is essentially public information that is integrity
verified and bound to the ciphertext as an added input to the encryption operation.  The integrity check occurs at decryption, which returns $\perp$ if
the integrity check on the labels fails.  This technique has been applied to several cryptographic applications under different names~\cite{Lim1993,Camenisch2003}, but has more
recently been used to boost the security of other schemes to be secure against chosen ciphertext attacks~\cite{Shoup2002,Camenisch2009,Hofheinz2013,Libert2014}.  
For example, the combination of a labeled PKE scheme with an IND-CCA secure PKE scheme can achieve CCA security with key-dependent messages.  
by including an additional one-time public key signature, where the verification key is attached as a label to ensure that it remains unmodified.  Another application for PLE labels is
in tagged signatures~\cite{Abe2013} and key encapsulation mechanisms~\cite{AGK08}.  In particular, the TKEM has been widely applied to allow for stronger security guarantees
in hybrid encryption schemes even when the security guarantees of the data encapsulation mechanism (DEM) are reduced.  In real-world protocol analysis, the
tags are frequently used to represent additional integrity verified public messages, such as the handshake messages in TLS~\cite{Jonsson2002}.  
More recently, Shibuya and Shikata~\cite{Shibuya2011}
developed a generalized definition called authenticated key encapsulation mechanism (AKEM), which captures the definition of TKEM as well as some related
constructions that do not clearly fit the TKEM definition.  However, because AKEM schemes allow the decryption to verify tags that are not public (e.g., the message
encrypted under the DEM), they do not technically fall under the standard definition of a labeled PKE scheme.