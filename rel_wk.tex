\subsection{Related Work}
\label{rel_wk}

\subsubsection{Deterministic PKE}

The first attempt to define public key security for fully deterministic schemes was by Bellare, Boldyreva, and O'Neill~\cite{BBO07}.
This work formalized a new definition of security that was achievable in the random oracle model as long as the min-entropy
of the plaintext space remained above a certain threshold.  Several other related definitions were proposed shortly after~\cite{BFOR08,BSO08},
producing instantiations in both the random oracle and the standard model, as well as constructions for semantically secure randomized schemes
and randomized key encapsulation mechanisms based on deterministic public key encryption.
Among all of these definitions, two major requirements persist: some restriction on the min-entropy of the plaintexts is always necessary, and
adversary access to the public key must be restricted through multi-phase adversaries. Raghunathan et al.~\cite{RSV13} proposed
a definition that relaxes this second restriction, but changes the adversary's goal to distinguishing between plaintext distributions
rather than individual plaintext queries.  While these deterministic definitions of security provide some insight into what is
required to maintain security in the presence of no randomness, they fail to demonstrate the potential for maintaining some
security when the randomness provided to the scheme is somewhere in between uniform and deterministic.  To answer
this question, hedged encryption schemes present definitions of security and encryption schemes that maintain some
amount of security even in the face of degrading quality randomness.  This is commonly achieved through harvesting randomness
from both the random coins provided to the encryption scheme and the plaintext distribution itself~\cite{BBNRTSSHY2009}.  
Recent work by Bellare and Hoang~\cite{BH15}
has produced standard model constructions for both deterministic and hedged encryption schemes by applying Universal Computational Extractors,
or UCEs~\cite{BHK13}.  This work also demonstrates the continuing utility of lossy trapdoor functions~\cite{PW11}, first applied to deterministic encryption
schemes by Boldyreva et al.~\cite{BSO08}.  Unfortunately, none of these definitions consider the potential for achieving equivalent security to 
semantic security given limited access to randomness, and can be viewed as special cases of nonce-based encryption by varying the
properties of the nonce.

\subsubsection{Password based encryption}