\paragraph{Applications. }\tsnote{Just a collection spot. Some things already incorporated into the body.  Doesn't make a lot of sense without first establishing some syntax though...}
\begin{itemize}
\item IoT: Are there settings in which the public key of a central server has been pre-deployed to many "users", but those users do not have reliable sources of randomness? I'm thinking IoT kinds of settings where a manufacturer may have installed a master public key on many embedded devices that do not have quality sources of randomness, but can keep a counter.  The counter need not be specifically for encryption, just something that could be co-opted for the purpose.  Perhaps there's no randomness source at all, but each device has a device ID that was burned in at the manufacturer, and thus can serve as a value with some amount of min-entropy.  In this case, we might have $\umd$ as the device ID (and $\auxinput$ as data sufficient to validate this ID), and $\rmd$ as the device state at the time plaintext~$M$ was sent.  When $(\rmd,M)$ are returned, the receiver can use the recovered~$\rmd$ to help make additional decisions about the reliability of~$M$ or the device itself.
%
\item Web applications: if the server already stores $Z=\langle \mathrm{salt},H(\mathrm{salt}\concat \mathrm{pwd})\rangle$ for a user, then this can be used (with $\umd$) to enable password-authentication.  Likewise, if the server is willing to keep some additional state for the user, then~$\pubiv$ can help to prevent replay.  (I wonder if IV-based PK-AEAD might be useful for reduced latency in resuming web sessions, etc.)
%
\item Traffic analysis: maybe one wants to hide the ordering of messages, i.e. "this is the third message in a stream", because this gives something away to an observer.  While a public counter/nonce would give this away, a secret counter/nonce would not.  So, maybe the public-IV~$\pubiv$ tells you that this is the third ciphertext in a stream, but the (recoverable) secret-AD~$\rmd$ tells you the actual ordinality of the plaintext.  This is one of the motivations, I believe, for the explicit public and secret message number in the call for CAESAR submissions.  (The idea appears implicitly in David McGrew's RFC 5116 "An Interface and Algorithms for Authenticated Encryption", which predates CAESAR.) 
%
\item Network stacks: it's possible, even likely in some settings, that the public-IV~$\pubiv$ and the secret-AD are provided by different entities.  The secret-AD may come from the application layer, perhaps without the application's user knowing it, while the public IV comes from the transport/network layer.  I'm not naming any particular application here, just pointing out that modern stacks may be doing this already in some ad-hoc fashion.
\item Onion/Garlic-routing schemes: ... ?

\end{itemize}
