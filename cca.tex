%%%%%%%%%%%%%%%%%%%%%%%%%%%%%%%%%%%%%%%%%%%%%%%%%%%%%%%%%%%%%%%%%%%%%%%%%%%%%%%%
\section{CCA Security}
\label{sec:cca}

Our primary security goals in the body exclude chosen-ciphertext attacks.
While in many applications of HE CCA attacks seem unlikely to be important, 
we have no evidence that they may not be security-relevant in some settings. 
We therefore formalize notions of message recovery security under CCAs, 
and show how to achieve it. 
%We also describe an ideal functionality based 
%notion of security for HE, and relate it to CCA under MR security.
%\tnote{Not sure we'll keep the ideal HE, but thought I'd dump this stuff in here
%for now.}


We modify the MR security notion from~\cite{JR14}
to include a decryption oracle in a standard way. See \figref{fig:mrcca-def}.
We define the MR-CCA advantage for an adversary $\advA$ against a scheme $\HEscheme$ by 
\bnm
\AdvMRCCA{\HEscheme,\mdist,\kdist}(\advA) = \Prob{\MRCCA_{\HEscheme,\mdist,\kdist}^\advA\Rightarrow \true} \;.
      %- \maxguess_\mdist
\enm
When working in the ROM, the game has an additional oracle $\Hash(\cdot)$ that associates to each
unique input message a returned string drawn randomly from $\bits^n$.  
Below we will work in the ideal cipher model, in which case the game has an additional 
pair of oracles implementing a random family of permutations $E(\cdot,\cdot)$ and their
inverses $E^{-1}(\cdot,\cdot)$. 
We let $q$ represent the number of queries by $\advA$ to the its oracles.
%where $\maxguess_\mdist = \max_{M} \Pr[M' = M \::\: M' \getm \mspace]$ reflects the probability
%of success when simply outputing the most
%likely point according to the distribution $\mdist$ over $\mspace$.

\begin{figure}[t]
\center
\fpage{.55}{
\hpages{.48}{
\underline{$\MRCCA_{\HEscheme,\mdist,\kdist}^\advA$}\\[2pt]
$\key^* \getk \kspace \next \msg^* \getm \mspace$\\
$\ctxt^* \getsr \encHE(\key^*,\msg^*)$\\
$\msg \getsr \advA^\decOracle(\ctxt^*)$\\
ret $\msg = \msg^*$\medskip
}{
\underline{subroutine $\decOracle(\ctxt)$}\\[2pt]
If $\ctxt = \ctxt^*$ then ret $\bot$\\
$\msg \gets \decHE(\key^*,\ctxt)$\\
ret $\msg$
}
}
\caption{Game used to define message recovery under chosen-ciphertext attack security.}
\label{fig:mrcca-def}
\end{figure}

It is easy to see that the hash-function based construction from \cite{JR14} does not meet
this notion of security. In particular, one need only flip a single bit of the ciphertext, 
query the decryption oracle, and from the response the adversary can determine the true 
plaintext precisely.

Let $\DTE$ be a DTE scheme with seed space $\sspace$.
Consider instead the following DTE-then-encrypt style scheme, $\HEscheme[\DTE,E]$ 
that uses an ideal cipher $E\Colon\bits^*\times\sspace\rightarrow\sspace$. 
It encrypts by encoding with the DTE to get a seed $\seed$, chooses a random $R$ of sufficient length, 
and then outputs $(R,E(K\concat R,\seed))$. The random value $R$ is only needed for achieving semantic security
when $K$ is large enough and we omit it in the following.
Then we have the following CCA analogue of \thref{thm:gen-cons}.

\begin{theorem}
\label{thm:gen-cons}
Fix distributions $\mdist,\kdist$, an encoding scheme $\DTE$ for $\mdist$ with seed space $\sspace$,
and the HE scheme $\HEscheme[\DTE,E]$ for ideal cipher $E\Colon\bits^*\times\sspace\rightarrow\sspace$. 
%and $\HEscheme[\DTE,\SEscheme]$ be the DTE-then-Encrypt scheme defined above. 
Let $\advA$ be an MR adversary against $\HEscheme$ making at most $q$ queries to its oracles. 
Then we give a specific adversary $\advB$ in the proof below such that 
%(1) $\AdvMR{\HEscheme,\mdist,\kdist}(\advA)
%    \;\le\; \AdvDTE{\DTE,\mdist}(\advB) \cdot \Ex{\load_{\HEscheme,\kdist}}$
%and
\bnm
  \AdvMRCCA{\HEscheme,\mdist,\kdist}(\advA)
    \;\le\; \AdvMR{\DTE,\mdist}(\advB) + \frac{q^2}{|\sspace|}
\enm
Adversary $\advB$ runs in time that of $\advA$ plus the time of at most $q$ samples from $\mdist$.
\end{theorem}


\begin{proof}[Sketch]
We first apply a standard birthday-bound argument to transition to a setting in which $E$ and $E^{-1}$ 
can be used with sampling with replacement. This accounts for the $q^2/|\sspace|$ term. In this world, now,
each query to $\decOracle$ uses an seeds chosen uniformly and independently of the one used by the challenge
encryption. (The attacker cannot query on the challenge ciphertext.) Thus the decryption oracle gives no
information about the challenge point, and so can be simulated by just returning draws from $\mdist$
\end{proof}





\iffalse

\tnote{Below needs to be revised... }
\paragraph{Ideal HE.}  Returning to the intuition given above, our goal 
for HE security is to ensure that trial decryptions result in plausible 
honeymessages. This should hold \emph{even} for low-entropy keys and 
messages, setting HE security apart from more traditional goals.
Informally, our security goal will be to show that encryption and decryption under
chosen message, ciphertext, and key attack is indistinguishable from such attacks against
an ideal honey encryption primitive. Towards this, we use an indifferentiability-style
security game in which an adversary interacts with one of two worlds. The first gives it
oracles including encryption, decryption, and an underlying ideal primitive $\prim$ (e.g., a RO or 
ideal cipher). The second ``ideal'' world gives it access to ideal HE encryption which always
returns random strings, ideal HE decryption which returns previously encrypted messages when
appropriate or, otherwise, returns a message sampled according to
$\mdist$, and a simulator $\simu$ that has access to the ideal encryption and decryption and
uses it to simulate $\prim$. Most importantly, the attacker is able to choose not only messages and
ciphertexts for its queries, but keys as well. 
A good HE scheme should be such that there exists a simulator for which no
adversary can distinguish between the two worlds.
%The latter is a pair of algorithms, the first 
%returns to each key and message input a new, random point. The second, upon input a 
%key and ciphertext, checks if that ciphertext was returned by the first algorithm previously
%and for that key, returning the associated message if so, and otherwise outputs a message
%randomly chosen according to $\mdist$. 
\figref{fig:sim-def} presents pseudocode for the two worlds.

\begin{figure}[t]
\hfpages{.2}{
\underline{proc.~$\OEnc(\key,\msg)$}:\\
$\ctxt \getsr \encHE^\PrimProc(\key,\msg)$\\
ret $\ctxt$\medskip

\underline{proc.~$\ODec(\key,\ctxt)$}:\\
$\msg \gets \decHE^\PrimProc(\key,\ctxt)$\\
ret $\msg$\medskip

\underline{proc.~$\PrimProc(X)$}\\
Ret $\prim(X)$
}{
\underline{proc.~$\IEnc(\key,\msg)$}:\\
$\ctxt \getsr \encHE(\key,\msg)$\\
$\ctxt \getsr \bits^{|\ctxt|}$\\
$\TabC[\key,\ctxt] \gets \msg$\\
ret $\ctxt$\medskip

\underline{proc.~$\IDec(\key,\ctxt)$}:\\
If $\TabC[\ctxt] \ne \bot$ then\\
\hspace*{1em} ret $\TabC[\key,\ctxt]$\\
$\msg \getm \mspace$\\
ret $\msg$\medskip

\underline{proc.~$\SimuProc(X)$}:\\
Ret $\simu^{\IEnc,\IDec}(X)$
}
\caption{Algorithms describing an ideal HE scheme.}
\label{fig:sim-def}
\end{figure}

For a given HE scheme $\HEscheme$, primitive $\prim$, and simulator $\simu$,  
we define the advantage of an adversary $\advA$ 
by:
\begin{multline*}
  \AdvHEDIST{\HEscheme,\prim,\simu}(\advA) = \\
        \Prob{\advA^{\OEnc,\ODec,\PrimProc} \Rightarrow 1} 
        - \Prob{\advA^{\IEnc,\IDec,\SimuProc} \Rightarrow 1}  \;.
\end{multline*}
We will consider both the random oracle model (ROM) and 
ideal cipher model (ICM). In the former case $\prim$ implements
a random function $\Hash$ (whose domain and range will be specified as appropriate) 
and in the latter case $\prim$ implements a pair of procedures $\Cipher$ and $\CipherInv$
that together implement a family of random permutations and their inverses.
%include a procedure implementing a random function or a pair of procedures
%implementing a family of ideal permutations and their inverses. 
%We denote the hash procedure by $\Hash$ and the ideal permutation procedures by
%$\Cipher$ and $\CipherInv$.  In this case $\encHE$ and $\decHE$ 
%are given oracle access to $\Hash$ or $\Cipher,\CipherInv$ as well. Correctness
%of the HE scheme must hold for any fixed coins of the oracles.


\paragraph{Ideal HE and message recovery.} We now show how ideal HE 
implies message recovery security. 
%Intuitively, we show that message
%recovery is as difficult with or without the target ciphertext, 
%since even brute-force key recovery doesn't help
%the adversary against an ideal HE scheme.  
%
We first must formalize message recovery security. Ideally one would like
message recovery to be impossible, meaning the probability of recovering a message
from a ciphertext is negligible. But this is only possible when both messages and
keys have high entropy, and here we have neither. We will therefore ask that, intuitively
speaking, that obtaining a ciphertext of the target message does not help the attacker in
learning the message. Formally, we define the MR security game as shown in \figref{fig:mr-def}
and define advantage for an adversary $A$ against a scheme $\HEscheme$ by 
\bnm
\AdvMR{\HEscheme,\mdist,\kdist}(\advA) = \Prob{\MR_{\HEscheme,\mdist,\kdist}^\advA\Rightarrow \true} \;.
      %- \maxguess_\mdist
\enm
%where $\maxguess_\mdist = \max_{M} \Pr[M' = M \::\: M' \getm \mspace]$ reflects the probability
%of success when simply outputing the most
%likely point according to the distribution $\mdist$ over $\mspace$.

\begin{figure}[t]
\center
\fpage{.25}{
\underline{$\MR_{\HEscheme,\mdist,\kdist}^\advA$}\\[2pt]
$\key^* \getk \kspace \next \msg^* \getm \mspace$\\
$\ctxt^* \getsr \encHE(\key^*,\msg^*)$\\
$\msg \getsr \advA(\ctxt^*)$\\
ret $\msg = \msg^*$
}
\caption{Game used to define message recovery security.}
\label{fig:mr-def}
\end{figure}

We have the following theorem. 
Fix a message space $\mspace$ and distribution $\mdist$. 
Then we let $\msg_1,\ldots,\msg_n$ be an ordering of $\mspace$ such 
that $\mdist(\msg_i) \ge \mdist(\msg_{i+1})$ for $1 \le i < n$.

\tnote{Ari suggested the below doesn't make sense: in particular,
one should just output an arbitrary message from the set. 
The intuition is that at this stage the attacker only gets 
a set of $\kappa$ iid samples from
the message distribution, so what else can he do?}

\begin{theorem}
Let $\mdist$ be a message distribution 
and $\kdist$ be a uniform key distribution over a set of $\kappa$ keys.
Let $\advA$ be an MR adversary in the ideal HE model for $\mdist$. 
Then
\bnm
    \AdvMR{\idealHE,\mdist,\kdist}(\advA) \le 
        \sum_{1 \le i \le n} \mdist(\msg_i) \cdot \gamma_i^{\kappa -1}
\enm
where $\gamma_i = \sum_{j \le i} \mu(\msg_j)$.
\end{theorem}

\begin{proof}
Let $\advA^*$ be the MR adversary that queries $\IDec(\key,\ctxt^*)$ for
each key $\key \in \kspace$. Let the resulting set of $\kappa$ messages 
be $G$. Then $\advA^*$ outputs the message $\msg \in G$ for which $\mdist(\msg)$
is highest. Adversary $\advA^*$ is optimal, meaning that no 
adversary $\advA$ achieves better advantage. To see this, first note that no 
adversary can increase advantage by making further queries; $\IEnc$ provides
no information and querying $\IDec$ on ciphertexts other than $\ctxt^*$ can
be replaced by fresh samples according to $\mdist$. Second outputing any
message other than the one 
\end{proof}

\begin{theorem}
Let $\mdist$ be a message distribution, $\HEscheme$ be an HE scheme built for $\mdist$,
and $\kdist$ be a key distribution. Let $\advA$ be an MR adversary against $\HEscheme$.
Then for any simulator $\simu$ there exists an adversary $\advB$ such that 
\bnm
  \AdvMR{\HEscheme,\mdist,\kdist,\prim}(\advA) \le \AdvHEDIST{\HEscheme,\simu,\prim}(\advB)
\enm
\end{theorem}

\tnote{Actually ok this theorem is wrong. Of course the attacker can just decrypt if
the keys are likely to be known. Sigh.}

\begin{proof}
Adversary $\advB$ works as follows. It chooses a key $\key^* \getk \kspace$ and 
a target message $\msg^*\getm\mspace$ and then queries $\encHE(\key^*,\msg^*)$ to
get back ciphertext $\ctxt^*$. It then runs $\advA^\PrimProc(\ctxt^*)$. Anytime
$\advA$ queries $\PrimProc$, adversary $\advB$ forwards the query to its own
$\PrimProc$. When $\advA$ finishes with output $\msg$, $\advB$ checks if $\msg = \msg^*$.
If so, it outputs $1$ (guessing it is in the real world) and otherwise outputs $0$. 

First, by construction we have that 
\bnm
  \Prob{\MR_{\HEscheme,\mdist,\kdist}^\advA\Rightarrow\true} = 
      \Prob{\advB^{\OEnc,\ODec,\PrimProc}\Rightarrow 1} \;.
\enm
We will next show that
\bne
\label{eqn:mr-indiff-proof}
  \maxguess(\mdist) \ge \Prob{\advB^{\IEnc,\IDec,\simu}\Rightarrow 1} \;.
\ene
Which, combined with the definitions of advantage for MR security and ideal HE 
security give the advantage statement of the theorem. Then to justify 
\eqref{eqn:mr-indiff-proof} we observe that $\advA$ recieves no information about $\msg^*$ 
during the course of its execution within $\advB^{\IEnc,\IDec,\simu}$. 
Thus $\advA$ can at best 
e now seek to lower bound the advantage of $\advB$. To do so we first note that


\end{proof}
\fi


